\documentclass{article}

% if you need to pass options to natbib, use, e.g.:
%     \PassOptionsToPackage{numbers, compress}{natbib}
% before loading neurips_2025

% The authors should use one of these tracks.
% Before accepting by the NeurIPS conference, select one of the options below.
% 0. "default" for submission
 \usepackage[preprint]{neurips_2025}
% the "default" option is equal to the "main" option, which is used for the Main Track with double-blind reviewing.
% 1. "main" option is used for the Main Track
%  \usepackage[main]{neurips_2025}
% 2. "position" option is used for the Position Paper Track
%  \usepackage[position]{neurips_2025}
% 3. "dandb" option is used for the Datasets & Benchmarks Track
 % \usepackage[dandb]{neurips_2025}
% 4. "creativeai" option is used for the Creative AI Track
%  \usepackage[creativeai]{neurips_2025}
% 5. "sglblindworkshop" option is used for the Workshop with single-blind reviewing
 % \usepackage[sglblindworkshop]{neurips_2025}
% 6. "dblblindworkshop" option is used for the Workshop with double-blind reviewing
%  \usepackage[dblblindworkshop]{neurips_2025}

% After being accepted, the authors should add "final" behind the track to compile a camera-ready version.
% 1. Main Track
 % \usepackage[main, final]{neurips_2025}
% 2. Position Paper Track
%  \usepackage[position, final]{neurips_2025}
% 3. Datasets & Benchmarks Track
 % \usepackage[dandb, final]{neurips_2025}
% 4. Creative AI Track
%  \usepackage[creativeai, final]{neurips_2025}
% 5. Workshop with single-blind reviewing
%  \usepackage[sglblindworkshop, final]{neurips_2025}
% 6. Workshop with double-blind reviewing
%  \usepackage[dblblindworkshop, final]{neurips_2025}
% Note. For the workshop paper template, both \title{} and \workshoptitle{} are required, with the former indicating the paper title shown in the title and the latter indicating the workshop title displayed in the footnote.
% For workshops (5., 6.), the authors should add the name of the workshop, "\workshoptitle" command is used to set the workshop title.
% \workshoptitle{WORKSHOP TITLE}

% "preprint" option is used for arXiv or other preprint submissions
 % \usepackage[preprint]{neurips_2025}

% to avoid loading the natbib package, add option nonatbib:
%    \usepackage[nonatbib]{neurips_2025}

\usepackage[utf8]{inputenc} % allow utf-8 input
\usepackage[T1]{fontenc}    % use 8-bit T1 fonts
\usepackage{hyperref}       % hyperlinks
\usepackage{url}            % simple URL typesetting
\usepackage{booktabs}       % professional-quality tables
\usepackage{amsfonts}       % blackboard math symbols
\usepackage{nicefrac}       % compact symbols for 1/2, etc.
\usepackage{microtype}      % microtypography
\usepackage{xcolor}         % colors



% Note. For the workshop paper template, both \title{} and \workshoptitle{} are required, with the former indicating the paper title shown in the title and the latter indicating the workshop title displayed in the footnote. 
\title{Deep Neural Networks with Gaussian Input}


% The \author macro works with any number of authors. There are two commands
% used to separate the names and addresses of multiple authors: \And and \AND.
%
% Using \And between authors leaves it to LaTeX to determine where to break the
% lines. Using \AND forces a line break at that point. So, if LaTeX puts 3 of 4
% authors names on the first line, and the last on the second line, try using
% \AND instead of \And before the third author name.


\author{%
  Simon Kuang and Xinfan Lin\thanks{\color{red} Use footnote for providing further information
    about author (webpage, alternative address)---\emph{not} for acknowledging
    funding agencies.} \\
  Department of Mechanical and Aerospace Engineering\\
  University of California, Davis\\
  Davis, CA 95616 \\
  \texttt{\{slku, lxflin\}@ucdavis.edu} \\
  % examples of more authors
  % \And
  % Coauthor \\
  % Affiliation \\
  % Address \\
  % \texttt{email} \\
  % \AND
  % Coauthor \\
  % Affiliation \\
  % Address \\
  % \texttt{email} \\
  % \And
  % Coauthor \\
  % Affiliation \\
  % Address \\
  % \texttt{email} \\
  % \And
  % Coauthor \\
  % Affiliation \\
  % Address \\
  % \texttt{email} \\
}


\usepackage{amsmath, amsthm, amssymb, mathtools, commath,
hyperref, cancel, bm}
% \usepackage{doi, xcolor}
% \usepackage[T1]{fontenc}

% \usepackage{eulervm}
% \usepackage[tracking, spacing]{microtype}
% \usepackage[margin=1in]{geometry}
\usepackage{algorithm,algpseudocode, algorithmicx}
\newtheorem{definition}{Definition}
\newtheorem{proposition}{Proposition}
\newtheorem{conjecture}{Conjecture}
\newtheorem{theorem}{Theorem}
\newtheorem{problem}{Problem}
\newtheorem{corollary}{Corollary}
\newtheorem{remark}{Remark}
\newtheorem{lemma}{Lemma}
\newtheorem{example}{Example}
\DeclareMathOperator{\trace}{\operatorname{tr}}
\DeclareMathOperator{\expect}{\mathbb{E}}
\DeclareMathOperator{\probability}{\mathbb{P}}
\DeclareMathOperator{\Var}{\operatorname{Var}}
\DeclareMathOperator{\Cov}{\operatorname{Cov}}

\DeclareMathOperator{\normal}{\mathrm N}
\DeclareMathOperator{\Normal}{\mathrm N^*}

% \setcitestyle{authoryear}
\begin{document}


\maketitle


\begin{abstract}
    The exact mean vector and variance-covariance matrix of a single-hidden-layer neural network (with the right choice of activation function) can be computed explicitly provided that the input is Gaussian.
    An extension to deep networks is surprisingly effective.
\end{abstract}


\section{Introduction}
A neural network, e.g.~one trained by supervised learning, maps a single input to a single output.
If the input is drawn from a probability distribution, then the output follows a pushforward distribution induced by the neural network.
The task of approximating the output distribution for a given input distribution is known as \emph{uncertainty quantification} (UQ).
It is motivated by the scientific interest in understanding adversarial input perturbations as well as the practical necessity of conducting inference in the presence of uncertainty.

We make the popular assumption that the input distribution is Gaussian with known mean and covariance.
Because neural networks are nonlinear, the output distribution generally does not have a tractable form.
Most methods including, our own, may be viewed as re-approximating each layer's neurons by a Gaussian distribution.

\begin{table}[h]
    \begin{center}
      \begin{tabular}{ll}
      \toprule
      Assumption & References \\
      \midrule
      \(\Sigma\) is small (linearized)  & \citet{titensky_uncertainty_2018, nagel_kalman-bucy-informed_2022}\\
      & \citet{petersen_uncertainty_2024, jungmann_analytical_2025} \\
      \(\Sigma\) is small (unscented) & \citet{astudillo_propagation_2011, abdelaziz_uncertainty_2015} \\
      \(\Sigma\) is diagonal          &  \citet{huber_bayesian_2020, wagner_kalman_2022,akgul_deterministic_2025}\\
      \(\mu = 0\)    & \citet{bibi_analytic_2018} \\
      \(\mu \to \infty\) & \citet{wu_deterministic_2019} \\
      no assumptions                 &  \citet{wright_analytic_2024}\\
      & this paper \\
      \bottomrule
      \end{tabular}
    \end{center}
    
    \caption{\label{tab:covariance-assumptions} Comparison of assumptions imposed on Gaussian approximations of neural network layers.}
\end{table}

\begin{table}[h]
    \begin{center}
      \begin{tabular}{ll}
      \toprule
      Activation function & References \\
      \midrule
      piecewise linear & \citet{bibi_analytic_2018,huber_bayesian_2020}\\
      & \citet{wright_analytic_2024,akgul_deterministic_2025} \\
      & \citet{wu_deterministic_2019} \\
      logistic (\(\approx\) piecewise exponential) &  \citet{astudillo_propagation_2011,abdelaziz_uncertainty_2015}\\
      logistic (\(\approx\) \(\Phi\)) & \citet{huber_bayesian_2020} \\
      Heaviside &  \citet{wu_deterministic_2019, wright_analytic_2024}\\
      GeLU & \citet{wright_analytic_2024}\\
      \(\Phi\) + affine & this paper \\
      \(\sin\) + affine & this paper
      \\      
      \bottomrule
      \end{tabular}
    \end{center}
    \caption{\label{tab:activation-functions} Activation functions for which moment propagation has been analyzed analytically.}
\end{table}

\textbf{Contribution.} We describe a general operator, valid for any activation function, for propagating a mean and covariance matrix through a single layer of a deep (residual) neural network, and compute \emph{exact} formulas for the probit and sine activation functions.
No further assumptions are made.
Numerical examples show orders-of-magnitude accuracy improvement over other simplifying assumptions.
Further numerical examples show surprisingly effective mean and covariance propagation for deep neural networks when the single-layer formula is chained over layers, and give a modest theoretical outlook for this phenomenon.

\textbf{Related work.}
Existing work on UQ for neural networks is classified by the assumptions imposed on the Gaussian distribution (Table~\ref{tab:covariance-assumptions}).
When the covariance matrix is small, the delta method justifies propagation of the mean by point evaluation and covariance by linearization \cite{titensky_uncertainty_2018,nagel_kalman-bucy-informed_2022,petersen_uncertainty_2024,jungmann_analytical_2025}, formulas with a long history \citep[Chapter 187]{gauss_theory_1857} \citep{taylor_introduction_1997}.
The Unscented Transformation, used in \citet{astudillo_propagation_2011,abdelaziz_uncertainty_2015}, is also justified by a Taylor (no relation) series argument \citep{julier_scaled_2002}.
Some other works \citep{huber_bayesian_2020,wagner_kalman_2022,akgul_deterministic_2025} make the mean-field assumption that each layers' neurons are independent, which can be a serviceable simplification for Bayesian learning, but leads to over-approximation of the covariance matrix.
These approximations can be made to look arbitrarily wrong by hand-constructed counterexamples {\color{red} appendices}.


Whether (and which) moments can be propagated analytically is ultimately an affordance of the activation function, reflected in the studies shown in Table~\ref{tab:activation-functions}.
The ReLU function \(\sigma(x) = \max(0, x)\) does not readily yield exact expressions for off-diagonal layer covariances:
\citet{bibi_analytic_2018} uses an analytical approximation around zero mean, and \citet{wu_deterministic_2019} uses an analytical approximation around infinite mean.
Neither does the traditional logistic function \(\sigma(x) = (1 + e^{-x})^{-1}\):
the works  \citet{astudillo_propagation_2011,abdelaziz_uncertainty_2015,huber_bayesian_2020} approximate the logistic activation function with another function having closed-form Gaussian moments.
For a general activation function, 
\citet{wright_analytic_2024} expands all intra-layer covariances as a power series (convergent for \(\rho \in (-1, 1)\)) in neuron-neuron correlation \(\rho\).

Our work attains the same generality and exactness as \citet{wright_analytic_2024}, but in terms of closed form formulas rather than power series.
Our layers are more general than all of the above works in that they superpose an affine transformation with the activation function, which allows analysis of residual networks and graphs \(x \mapsto (x, f(x))\).

\section{Problem formulation}
\subsection{Notation}
The activation function of a neural network is denoted \(\sigma:\mathbb R \to \mathbb R\) and understood to be applied elementwise.
Except for parameters \(A, b, C, d\), capital letters refer to random variables.
The layers of a neural network are indicated by superscripts, e.g.~\(A^k\) is a matrix of parameters for the \(k\)th layer.

If \(X\) is a square-integrable random vector, the notation \(\normal X\) refers to a random variable having distribution \(\mathcal N(\expect X, \Cov X)\).
If \(f\) is a neural network, the notation \(\Normal f(X)\) is a Normal random variable in which the normality approximation is applied to each layer, in a sense we will make precise in the next sections.

\subsection{Conventions}
Let \(\sigma : \mathbb R \to \mathbb R\) be an activation function.

\begin{definition}
    \label{def:layer-function}
    A layer function is a function \(g:\mathbb R^n \times \mathbb R^{m \times n} \times \mathbb R^m \times \mathbb R^{m \times n} \times \mathbb R^m \to \mathbb R^m\) defined by \(g(x; A, b, C, d) = \sigma(A x + b) + C x + d\), where \(A \in \mathbb R^{m \times n}, b \in \mathbb R^m, C \in \mathbb R^{m \times n}, d \in \mathbb R^m\) are parameters.
\end{definition}

We will often omit the ``type annotations'' indicating the input and output dimensions when it is only necessary to imply that they are compatible.

\begin{definition}
    A neural network with \(\ell \) layers is the function \(f: \mathbb R^{n_x} \to \mathbb R^{n_y}\) defined by
    \begin{align*}
        f(x) &= f^\ell(x) \\
        f^k(x) &= g(f^{k-1}(x); A^k, b^k, C^k, d^k) & \forall k &\in \cbr{1 \ldots \ell} \\
        f^0(x) &= x
    \end{align*}
\end{definition}

\subsection{Problem}
\label{sec:problem}
\begin{problem}
    Let \(f\) be a neural network with \(\ell\) layers.
    Given \(X \sim \mathcal N(\mu, \Sigma)\), characterize the distribution of \(Y_0 = f(X)\).
\end{problem}

Because this problem is intractable in general, our approach is to replace it with the proxy:

\begin{definition}
    Let \(f\) be a neural network with \(\ell\) layers.
    Given \(X \sim \mathcal N(\mu, \Sigma)\), the layer-wise Gaussian approximation of \(f(X)\), denoted \(Y = \Normal f(X)\), is the random variable such that
    \begin{align*}
        Y &=  Y^\ell\\
        Y^k &= \normal g(Y^{k-1}; A^k, b^k, C^k, d^k) & \forall k &\in \cbr{1 \ldots \ell} \\
        Y^0 &= X
    \end{align*}
\end{definition}

In Appendix~\ref{app:theoretical-guarantees}, we give a theoretical bound on the dissimilarity between the laws of \(Y_0\) and \(Y\).

An auxiliary distribution used for benchmarking is:
\begin{definition}
    Let \(f \) be a neural network with \(\ell\) layers.
    Given \(X \sim \mathcal N(\mu, \Sigma)\), the pseudo-true Gaussian approximation of \(f(X)\) is \(Y_1 = \normal f(X)\).
\end{definition}

\section{Methods}
It turns out that only three transcendental functions are needed to compute the first two moments of the pushforward measure of a Gaussian distribution under a layer defined by Def.~\ref{def:layer-function}.
\begin{definition}
  \label{def:moment-maps}
  Given a nonlinear function \(\sigma: \mathbb{R} \to \mathbb R\),
  the functions \(M_\sigma: \mathbb{R}  \times \mathbb{R}_+ \to
  \mathbb{R}\) and \(K_\sigma, L_\sigma: \mathbb{R}^2 \times \mathbb
  R_+^2 \times [0, 1] \to \mathbb{R}\) are
  \begin{align*}
    M_\sigma(\mu; \nu) &= \expect{\sigma(X)},
    & X &\sim \mathcal N(\mu, \nu)
    \\
    K_\sigma(\mu_1, \mu_2; \nu_{11}, \nu_{22}, \nu_{12}) &= \Cov
    (\sigma(X_1), \sigma(X_2)),
    &
    \begin{pmatrix}
      X_1 \\ X_2
    \end{pmatrix} &\sim \mathcal N\del{
      \begin{pmatrix}
        \mu_1 \\ \mu_2
      \end{pmatrix},
      \begin{pmatrix}
        \nu_{11} & \nu_{12} \\
        \nu_{12} & \nu_{22}
      \end{pmatrix}
    }
    \\
    L_\sigma(\mu_1; \nu_{11}, \nu_{22}, \nu_{12}) &= \Cov
    (\sigma(X_1), X_2),
    &
    \begin{pmatrix}
      X_1 \\ X_2
    \end{pmatrix} &\sim \mathcal N\del{
      \begin{pmatrix}
        \mu_1 \\ \star
      \end{pmatrix},
      \begin{pmatrix}
        \nu_{11} & \nu_{12} \\
        \nu_{12} & \nu_{22}
      \end{pmatrix}
    }
    \\
  \end{align*}
\end{definition}

\begin{lemma}
  Let \(g\) be the function defined by \(g(x; A, b, C, d) = \sigma(A
  x+ b) + C x + d\).
  Let \(X \sim \mathcal N(\mu, \Sigma)\).
  Then
  \begin{align*}
    \del{\expect g(X; A, b, C, d)}_i &= M_\sigma(\mu_i; \nu_{ii}) +
    (C\mu)_i + d_i
  \end{align*}
  and
  \begin{align*}
    \begin{split}
      \del{\Cov g(X; A, b, C, d)}_{i, j} &=
        K_\sigma\del{
          \mu_i, \mu_j; \nu_{ii}, \nu_{jj}, \nu_{ij}
        } \\
        &\quad + L_\sigma\del{
          \mu_i; \nu_{ii}, \tau_{jj},\kappa_{ij}
        }
        + L_\sigma\del{
          \mu_j; \nu_{jj}, \tau_{ii},\kappa_{ji}
        } \\
        &\quad + \tau_{ij}.
    \end{split}
  \end{align*}
  where for all valid indices \((i, j)\),
  \begin{align*}
    \mu_i &= (A\mu + b)_i
    &
    \tau_{ij}
    &= (C\Sigma C^\intercal)_{i,j}
    \\
    \nu_{ij} &= (A\Sigma A^\intercal)_{i,j}
    &
    \kappa_{ij}
    &= (A \Sigma C^\intercal)_{i,j}
  \end{align*}
\end{lemma}
These are five-variable transcendental functions that are too complex to be represented by a look-up table, and thus need to be computed analytically for a strategic choice of activation function \(\sigma\).
We work them out for the probit sigmoid activation in App.~\ref{app:probit} and for the sine activation in App.~\ref{app:sine}.

\section{Numerical Results}
\subsection{Interpolation regime}
\label{sec:interpolation-regime}
Much of the effectiveness of overparameterized (wide and deep) neural networks is now understood by reference to an ``interpolation regime'' in which a neural network is supervised to fit training data exactly while simultaneously minimizing a regularization norm \citep{muthukumar_harmless_2020,schaeffer_double_2023,belkin_reconciling_2019}.
In this section, we train a neural network \(f\) with 121,201 parameters to interpolate \(y = f(x)\) for twenty pairs \((x, y)\), where we have randomly drawn \((x, y) \sim \mathcal N(0_2, I_2)\).
Details on the training process are provided in App.~\ref{app:interpolation-regime}.
The training data and the resulting function \(f\) are shown in Fig.~\ref{fig:interpolation-regime-function}.

The input is \(X \sim \mathcal N(0, 1)\).
The distributions \(Y\), \(Y_0\), and \(Y_1\) follow \S\ref{sec:problem}.
Our approximation beats the other approximations by a mile on all counts: closest mean to \(Y_1\), closest variance to \(Y_1\), least Wasserstein distance to \(Y_0\), and least Kullback-Leibler divergence to \(Y_1\) (Table~\ref{tab:interpolated-nominal-variance}).
These figures are consistent with inspection of the probability distributions (Fig.~\ref{fig:interpolated-regime-pdf}).

Details on the Monte Carlo simulation method and the summary statistics are given in App.~\ref{app:monte-carlo}.

\begin{table}
  \begin{center}
    \input{tables/interpolated-nominal-variance}
  \end{center}
  \caption{\label{tab:interpolated-nominal-variance} Normal distributions arising from the problem in \S\ref{sec:interpolation-regime}.}
\end{table}

\begin{figure}
  \centering
  \includegraphics[]{figures/interpolated-function.pdf}
  \caption{Training data (dots) and learned function (solid line) in the interpolation regime.}
  \label{fig:interpolation-regime-function}
\end{figure}

\begin{figure}
  \centering
  \includegraphics[]{figures/interpolated-nominal-variance-density.pdf}
  \caption{\label{fig:interpolated-regime-pdf} Histogram of \(Y_0\) and probability density functions of its approximants in \S\ref{sec:interpolation-regime}.}
\end{figure}

\section{Computational complexity}
\subsection{Theoretical}

\subsection{Empirical}

\begin{ack}
Use unnumbered first level headings for the acknowledgments. All acknowledgments
go at the end of the paper before the list of references. Moreover, you are required to declare
funding (financial activities supporting the submitted work) and competing interests (related financial activities outside the submitted work).
More information about this disclosure can be found at: \url{https://neurips.cc/Conferences/2025/PaperInformation/FundingDisclosure}.


Do {\bf not} include this section in the anonymized submission, only in the final paper. You can use the \texttt{ack} environment provided in the style file to automatically hide this section in the anonymized submission.
\end{ack}

\bibliographystyle{abbrvnat}
\bibliography{nn-filtering.bib}


%%%%%%%%%%%%%%%%%%%%%%%%%%%%%%%%%%%%%%%%%%%%%%%%%%%%%%%%%%%%

\appendix

\section{Derivation of uncertainty propagation formulas for probit activation}
\label{app:probit}
In this appendix, we derive the \(M_\sigma\), \(K_\sigma\), and
\(L_\sigma\) functions (Def.~\ref{def:moment-maps}) for the normal
CDF activation function
\begin{align}
  \sigma(x) &= \sqrt{\frac{2}{\pi}} \int_{0}^x e^{-\frac{1}{2} u^2}
  du = 2 \Phi(x) - 1, \quad \Phi(x) = \probability_{Z \sim \mathcal
  N(0, 1)}(Z \leq x)
  \label{eq:activation}
\end{align}

\begin{definition}
  The bivariate normal CDF \(\Phi_2\) is defined by
  \begin{align*}
    \Phi_2(h, k; \rho)
    &= \probability\sbr{Z_1 \leq h, Z_2\leq k},
    \intertext{where}
    \begin{pmatrix}
      Z_1 \\ Z_2
    \end{pmatrix}
    &\sim \mathcal N\del{\begin{pmatrix} 0 \\ 0 \end{pmatrix}, \begin{pmatrix}
      1 & \rho \\ \rho & 1
    \end{pmatrix}}.
  \end{align*}
\end{definition}
\begin{lemma}
  Let \(M_\sigma\) be defined as in Def.~\ref{def:moment-maps}, with
  \(\sigma\) as in \eqref{eq:activation}.
  \begin{align*}
    M_\sigma(\mu; \nu) &= \sigma\del{\frac{\mu}{\sqrt{1 + \nu}}}
  \end{align*}
  \label{lem:mean}
\end{lemma}
\begin{proof}
  Let \(X \sim \mathcal N(\mu, \nu)\).
  \begin{align}
    \expect \sigma(X)
    &= -1+2\expect \Phi\del{X}
    \\
    &= -1+2\expect \probability \sbr{{Z \leq X} \mid X}
    \tag{introducing an independent \(Z \sim \mathcal{N}(0, 1)\)}
    \\
    &= -1+2\probability\sbr{{Z \leq X}}
    \tag{by the law of total probability}
    \\
    &= -1+2\probability \sbr{Z - X \leq 0}
  \end{align}
  We conclude by noting that the random variable \(Z - X\) has a
  Normal distribution with mean \(-\mu\) and variance \(1 + \nu^2\).
\end{proof}

\begin{lemma}
  Let \(K_\sigma\) be defined as in Def.~\ref{def:moment-maps}, with
  \(\sigma\) as in \eqref{eq:activation}.
  \begin{align*}
    K(\mu_1, \mu_2; \nu_{11}, \nu_{22}, \nu_{12}) &=
    4 \left.
    % \sbr{
    \Phi_2\del{
      \frac{\mu_1}{\sqrt{1 + \nu_{11}}},
      \frac{\mu_2}{\sqrt{1 + \nu_{22}}};
      \rho'
    }
    % }
    \right|^{\rho' = \frac{\nu_{12}}{\sqrt{(1 +
    \nu_{11})(1 + \nu_{22})}}}_{\rho' = 0},
  \end{align*}
  where \(\Phi_2\) is the bivariate normal CDF.
\end{lemma}
\begin{proof}
  Let
  \begin{align}
  \begin{pmatrix} X_1 \\ X_2 \end{pmatrix}
   \sim \mathcal N\left(\begin{pmatrix}
    \mu_1
    \\
    \mu_2
  \end{pmatrix},
  \begin{pmatrix}
    \nu_{11}
    &
    \nu_{12}
    \\
    \nu_{12}
    &
    \nu_{22}
  \end{pmatrix}\right).
  \end{align}
  By a similar calculation as in Lemma~\ref{lem:mean}, we introduce independent
  \(Z_1, Z_2 \sim \mathcal N(0, 1)\) and have
  \begin{align}
    \Cov (\sigma(X_1), \sigma(X_2))
    &= 4 \Cov (\Phi(X_1), \Phi(X_2))
    \\
    &= 4\expect \Phi(X_1) \Phi (X_2) - 4\expect \Phi(X_1) \expect \Phi(X_2)
    \\
    &= 4 \probability\sbr{Z_1 \leq X_1, Z_2 \leq X_2} -
    4\probability\sbr{Z_1 \leq X_1} \probability\sbr{Z_2 \leq X_2}
    \\
    &= 4 \probability\sbr{Z_1 - X_1 \leq 0, Z_2 - X_2 \leq 0} -
    4\probability\sbr{Z_1- X_1\leq 0} \probability\sbr{Z_2 - X_2 \leq 0}.
  \end{align}
  We conclude by using the fact that \((Z_1 - X_1, Z_2 - X_2)\) is
  jointly Normal with distribution
  \begin{align}
    \begin{pmatrix}
      Z_1 - X_1
      \\
      Z_2 - X_1
    \end{pmatrix}
    \sim
    \mathcal{N}\del{
      \begin{pmatrix}
        -\mu_1
        \\
        -\mu_2
      \end{pmatrix},
      \begin{pmatrix}
        1 + \nu_{11}
        &
        \nu_{12}
        \\
        \nu_{12}
        &
        1 + \nu_{22}
      \end{pmatrix}
    }.
  \end{align}
\end{proof}
\begin{remark}
  The bivariate normal CDF can be a difficult transcendental function
  to evaluate.
  Many software packages such as SciPy \citep{wagner_kalman_2022}
  implement the multivariate normal CDF by a (quasi-) Monte Carlo
  integration over \(\mathbb{R}^n\), which is too expensive for our purposes.
  Furthermore, the expression \(\Phi(h, k; \rho) - \Phi(h, k; 0)\) is
  vulnerable to cancellation error for extreme values of \(h\),
  \(k\), and \(\rho\).
  To avoid this, we implement \(K\) by 10-point Gaussian quadrature
  of the one-dimensional proper integral
  \begin{align*}
    \Phi_2(h, k; \rho) -
    \Phi_2(h, k; 0)
    &= \int_0^\rho \partial_\rho' \Phi_2(h, k; \rho') \dif\rho',
  \end{align*}
  using
  \begin{align*}
    \partial_\rho \Phi_2(h, k; \rho)
    &= \frac{1}{2\pi \sqrt{1 - \rho^2}} e^{-\frac{1}{2(1 - \rho^2)}
    \del{h^2 + k^2 - 2 \rho h k}},
  \end{align*}
  a helpful identity found in \citet{drezner_computation_1990}
\end{remark}


\begin{lemma}[Stein's lemma]
  \label{lem:stein}
  Let \(L_\sigma\) be defined as in Definition \ref{def:moment-maps}, and suppose that \(\sigma\) is differentiable.
  Then
  \begin{align*}
    L_\sigma(\mu_1, \mu_2; \nu_1^2, \nu_2^2, \rho) &= \rho
    \nu_1 \nu_2 \expect \sigma'(Z_1),
    & Z_1 &\sim \mathcal N(\mu_1, \nu^2_1).
  \end{align*}
\end{lemma}


\begin{lemma}
  Let \(L_\sigma\) be defined as in Def.~\ref{def:moment-maps}, with
  \(\sigma\) as in \eqref{eq:activation}.
  Then
  \begin{align*}
    L_\sigma(\mu_1, \mu_2; \nu_{11}, \nu_{22}, \nu_{12}) &= 2
    \frac{\nu_{12}}{\sqrt{1 + \nu_{11}}}
    \phi\del{\frac{\mu_1}{\sqrt{1 + \nu_{11}}}}.
  \end{align*}
\end{lemma}
\begin{proof}
   Let
  \begin{align}
  \begin{pmatrix} X_1 \\ X_2 \end{pmatrix}
   \sim \mathcal N\left(\begin{pmatrix}
    \mu_1
    \\
    \mu_2
  \end{pmatrix},
  \begin{pmatrix}
    \nu_{11}
    &
    \nu_{12}
    \\
    \nu_{12}
    &
    \nu_{22}
  \end{pmatrix}\right).
  \end{align}
  Using Lemma \ref{lem:stein}, we have
  \begin{align}
    L_\sigma(\mu_1, \mu_2; \nu_{11}, \nu_{22}, \nu_{12})
    &= \nu_{12} \expect \sigma'(X_1)
    \\
    &= 2\nu_{12} \expect \phi(X_1)
  \end{align}
  where \(\phi = \Phi'\).
  We conclude by appealing to the Gaussian integral identity that
  when \(Z \sim \mathcal N(\mu_Z, \nu_Z)\),
  \begin{align}
    \expect \phi(Z) = \frac{1}{\sqrt{1 + \nu_Z}}
    \phi\del{\frac{\mu_Z}{\sqrt{1 + \nu_Z}}}.
  \end{align}
\end{proof}

\section{Derivation of uncertainty propagation formulas for sine activation}
\label{app:sine}
In this appendix, we derive the \(M_\sigma\), \(K_\sigma\) and
\(L_\sigma\) functions (Def.~\ref{def:moment-maps}) for the sinusoidal activation function
\begin{align}
  \label{eq:activation-sin}
  \sigma(x) &= \sin(x).
\end{align}
We begin by recalling the identities that if \(Z \sim \mathcal N(\mu, \nu)\),
\begin{align}
\label{eq:expect-sin}
  \expect \sin(Z) &= e^{-\nu/2} \sin(\mu)
  \\
  \label{eq:expect-cos}
  \expect \cos(Z) &= e^{-\nu/2} \cos(\mu)
\end{align}

From \eqref{eq:expect-sin} immediately follows:
\begin{lemma}
  Let \(M_\sigma\) be defined as in Def.~\ref{def:moment-maps}, with
  \(\sigma\) as in \eqref{eq:activation-sin}.
  Then
  \begin{align*}
    M_\sigma(\mu; \nu) &= e^{-\nu/2}\sin(\mu).
  \end{align*}
\end{lemma}

\begin{lemma}
  Let \(K_\sigma\) be defined as in Def.~\ref{def:moment-maps}, with
  \(\sigma\) as in \eqref{eq:activation-sin}.
  Then
  \begin{align*}
    \begin{split}
      K_\sigma(\mu_1, \mu_2; \nu_{11}, \nu_{22}, \nu_{12}) 
      &= \frac{1}{2} \sbr{e^{\nu^* + \nu_{12}} - e^{\nu^*}} \cos(\mu_1 - \mu_2) \\
      &\quad - \frac{1}{2} \sbr{e^{\nu^* - \nu_{12}} - e^{\nu^*}} \cos(\mu_1 + \mu_2),
    \end{split}
    \intertext{where}
    \nu^* &= -\frac{\nu_{11} + \nu_{22}}{2}.
  \end{align*}
\end{lemma}
\begin{proof}
  Let
  \begin{align}
  \begin{pmatrix} X_1 \\ X_2 \end{pmatrix}
   \sim \mathcal N\left(\begin{pmatrix}
    \mu_1
    \\
    \mu_2
  \end{pmatrix},
  \begin{pmatrix}
    \nu_{11}
    &
    \nu_{12}
    \\
    \nu_{12}
    &
    \nu_{22}
  \end{pmatrix}\right).
  \end{align}
  Then by inserting \eqref{eq:activation-sin}, we have
  \begin{align}
    \Cov(\sigma(X_1), \sigma(X_2))
    &= \expect \sin(X_1) \sin(X_2) - \expect \sin(X_1) \expect \sin(X_2).
    \label{eq:cov-sin-sin}
  \end{align}
  % https://en.wikipedia.org/wiki/List_of_trigonometric_identities#Product-to-sum_and_sum-to-product_identities
  Using some trigonometric identities and \eqref{eq:expect-cos}, the first term of \eqref{eq:cov-sin-sin} becomes
  \begin{align}
    \expect \sin(X_1) \sin(X_2)
    &= \frac{1}{2} \expect \cos(\underbrace{X_1 - X_2}_{
      \mathcal N(\mu_1 - \mu_2, \nu_{11} + \nu_{22} - 2\nu_{12})
      })
    - \frac{1}{2} \expect \cos(\underbrace{X_1 + X_2}_{\mathcal N(\mu_1 + \mu_2, \nu_{11} + \nu_{22} + 2\nu_{12})})
    \\
    \begin{split}
      &= \frac{1}{2} \exp\del{-\frac{\nu_{11} + \nu_{22}}{2} + \nu_{12}} \cos(\mu_1 - \mu_2) \\
      &\quad - \frac{1}{2} \exp\del{ -\frac{\nu_{11} + \nu_{22}}{2} - \nu_{12}} \cos(\mu_1 + \mu_2)
    \end{split}
  \end{align}
  The second term of \eqref{eq:cov-sin-sin} becomes
  \begin{align}
    \begin{split}
      \expect \sin(X_1) \expect \sin(X_2)
      &= \frac{1}{2} \exp\del{-\frac{\nu_{11} + \nu_{22}}{2}} \cos(\mu_1 - \mu_2) \\
      &\quad - \frac{1}{2} \exp\del{ -\frac{\nu_{11} + \nu_{22}}{2}} \cos(\mu_1 + \mu_2)
    \end{split}
  \end{align}
  The result follows from collecting like terms.
\end{proof}

\begin{lemma}
  Let \(L_\sigma\) be defined as in Def.~\ref{def:moment-maps}, with
  \(\sigma\) as in \eqref{eq:activation-sin}.
  Then
  \begin{align*}
    L_\sigma(\mu_1, \mu_2; \nu_{11}, \nu_{22}, \nu_{12})
    &=
    \nu_{12} e^{-\nu_{11}/2} \cos(\mu_1). 
  \end{align*}
\end{lemma}
\begin{proof}
   Let
  \begin{align}
  \begin{pmatrix} X_1 \\ X_2 \end{pmatrix}
   \sim \mathcal N\left(\begin{pmatrix}
    \mu_1
    \\
    \mu_2
  \end{pmatrix},
  \begin{pmatrix}
    \nu_{11}
    &
    \nu_{12}
    \\
    \nu_{12}
    &
    \nu_{22}
  \end{pmatrix}\right).
  \end{align}
  Using Lemma \ref{lem:stein}, we have
  \begin{align}
    L_\sigma(\mu_1, \mu_2; \nu_{11}, \nu_{22}, \nu_{12})
    &= \nu_{12} \expect \sigma'(X_1)
    \\
    &= \nu_{12} \expect \cos(X_1)
    \\
    &= \nu_{12} e^{-\nu_{11}/2} \cos(\mu_1)
  \end{align}
  by \eqref{eq:expect-cos}.
\end{proof}

\section{Theoretical Guarantees}
\label{app:theoretical-guarantees}
The objective of this section is to provide a theoretical analysis of the dissimilarity between \(Y_0\) and \(Y\).
We recall the layer-by-layer Normal approximation resulting in \(Y\) alongside the exact neural network formula resulting in \(Y_0\).
\begin{subequations}
  \begin{align}
    Y &= Y^\ell & Y_0 &= Y^\ell_0
    \\
    Y^k &= g^k(Y^{k-1}), & Y_0^k &= g^k(Y_0^{k-1}), & k \in
    \cbr{1 \ldots \ell},
    \label{eq:hidden-layer-comparison}
    \\
    Y^0 &= X & Y_0^0 &= X,
    \intertext{where \(g^k\) is the function}
    g^k(x) &= g(x; A^k, b^k, C^k, d^k).
\end{align}
\end{subequations}
We will use the Wasserstein distance to measure the distance between \(Y\) and \(Y_0\).

\begin{definition}
Let \(X\) and \(Y\) be random variables taking values  \(\mathbb{R}^n\).
The Wasserstein distance \(d_\text{W} (\mu, \nu)\) is 
\begin{align*}
  d_\mathrm{W}(X, Y) &= \sup_{\left\|\nabla h\right\|_{\infty} \leq 1} \expect (h(X) - h(Y)).
\end{align*}
\end{definition}

The ultimate goal is to show that \(d_\text{W} (Y_0,Y)\) is small.
We will build up this quantity by recursion through the layers of the neural network.
The key idea is that the error induced by Normal approximation gets worse with every subsequent layer of the network.
% Heuristically, if \(X\) is Normal, then \(Y_0^0\) is close to Normal and therefore well-approximated by its first and second moments.
% Then \(Y_0^1\) is a less Normal than \(Y_0^0\), and so on.
% Thus, roughly speaking, the approximation error in distribution should, by a triangle inequality, scale rougly linearly in the depth of the network.

\subsection{Recursive triangle inequality}
Our basic triangle inequality is
\begin{align}
  \Delta^k &:= d_\mathrm{W}(Y_0^k, Y^k) 
  \label{eq:discrepancy}
  \\
  &= d_\mathrm{W}(g^k(Y_0^{k-1}), Y^k)
  \tag{definition of hidden layer \eqref{eq:hidden-layer-comparison}}
  \\
  &\leq d_\mathrm{W}(g^k(Y_0^{k-1}), g^k(Y^{k-1})) + d_\mathrm{W}(g^k(Y^{k-1}), Y^k)
  \tag{triangle inequality}
  \\
  &\leq \left\|\nabla g^k\right\|_\infty d_\mathrm{W}(Y_0^{k-1}, Y^{k-1}) + d_\mathrm{W}(g^k(Y^{k-1}), Y^k)
  \tag{Lipschitz property of \(d_\mathrm{W}\)}
  \\
  &\leq \left\|\nabla g^k\right\|_\infty \Delta^{k-1} + d_\mathrm{W}(g^k(Y^{k-1}), Y^k)
  \label{eq:discrepancy-recursion}
  % \tag{recursion to \eqref{eq:discrepancy}}
\end{align}
In words, at a hidden layer \(k\),
\begin{multline}
  \text{error at layer \(k\)}
  = \del{\text{Lipschitz constant of layer \(k\)}} \del{\text{error at layer \(k-1\)}} 
  \\
  + \del{\text{normal approximation error at layer \(k\)}}  
\end{multline}
If \(X\) is Gaussian, then \(\Delta^0 = 0\).
We need to fill two blanks to make this formula concrete.

\subsection{Lipschitz constant}
First, the Lipschitz constant \(\left\|\nabla g^k\right\|_\infty\):
\begin{align}
  \nabla g(x; A, b, C, d)
  &= \nabla \sbr{\sigma(A x + b) + C x +  d}
  \\
  &= 2A^\intercal \operatorname{diag} \cbr{\sigma'(Ax + b)} + C^\intercal
  \intertext{To bound this quantity,}
  \left\|\nabla g(x; A, b, C, d)
  \right\|
  &\leq \sup_{x} \left\|2 A^\intercal \operatorname{diag} \cbr{\sigma'(Ax + b)} + C^\intercal\right\|
  \\
  &\leq \left\|\sigma'\right\|_\infty \left\|A\right\|
  + \left\|C\right\|
\end{align}

\subsection{Non-normality}
Second, we need to bound the distance between \(g^k(Y^{k-1})\) and \(Y^k\).
The importance of the recursion \eqref{eq:discrepancy-recursion} is that at layer \(k\), we assess the goodness of fit between \(g^k(Y^{k-1})\), a nonlinear transformation of a Normal random vector, and \(Y^k\), its approximant.
The input to the nonlinearity is Normal.
Since \(Y^{k-1}\) is Normal, we can use techniques from the functional analysis of Gaussian spaces, such as this second-order Poincar\'e inequality:
\begin{theorem}[Second-order Poincar\'e inequality {\cite[Theorem~7.1]{nourdin_second_2009}}]
Suppose that \(Y = f(X)\) is a random vector taking values in \(\mathbb{R}^n\), where \(X\) is a standard Normal random vector, and that \(Y\) has mean \(\mu\) and covariance matrix \(\Sigma\).
Then
\begin{align*}
  d_\mathrm{W}\del{Y, \mathcal N(\mu, \Sigma)}
  \leq \frac{3}{\sqrt 2} \|\Sigma^{-1}\| \|\Sigma\|^{1/2}
  \sbr{\sum_{i = 1}^n \del{\expect \left\|D^2 f_i\right\|^4}^{1/4}}
  \sbr{\sum_{i = 1}^n \del{\expect \left\|D f_i\right\|^4}^{1/4}}
\end{align*}
\end{theorem}


\section{Training details for \S\ref{sec:interpolation-regime}}
\label{app:interpolation-regime}

\subsection{Network Architecture}
The neural network is implemented using JAX and Equinox, with the following architecture:
\begin{itemize}
    \item Input layer: 1D input (scalar)
    \item First hidden layer: 200 neurons with sinusoidal activation ($\sigma(x) = \sin(x)$)
    \item Three residual hidden layers: 200 neurons each with sinusoidal activation
    \item Output layer: Linear layer with 1D output (scalar)
\end{itemize}

\subsection{Training Details}
The training process follows these specifications:
\begin{itemize}
    \item Training samples: 10 pairs $(x, y)$ where $x, y \sim \mathcal{N}(0, 1)$
    \item Loss function: Mean squared error (MSE) between predicted and target outputs
    \item Optimization: AdamW optimizer with learning rate $10^{-5}$
    \item Training duration: 10,000 iterations
    \item Batch size: Full batch (all 10 samples used in each update)
    \item Weight initialization:
    \begin{itemize}
        \item First layer: Zero matrix for weights, uniform random for biases
        \item Hidden layers: Random orthogonal projection for weights, uniform random for biases
        \item Output layer: Zero initialization for both weights and biases
    \end{itemize}
\end{itemize}

\subsection{Implementation Details}
The implementation uses JAX's just-in-time compilation and automatic differentiation capabilities:
\begin{itemize}
    \item Uses Equinox for neural network implementation and training utilities
    \item Employs JIT compilation for the training step to improve performance
    \item Implements batched computation using JAX's vectorization capabilities
    \item Training is performed on CPU (though the code supports GPU acceleration)
\end{itemize}

\subsection{Training Process}
The network is trained to minimize the mean squared error between predictions and targets:
\[
\frac{1}{N}\sum_{i=1}^N (f_\theta(x_i) - y_i)^2
\]
where $f_\theta$ represents the neural network with parameters $\theta$, and $(x_i, y_i)$ are the training samples. The training process uses the full dataset for each update.

\section{Monte Carlo simulation and analysis}
\label{app:monte-carlo}

%%%%%%%%%%%%%%%%%%%%%%%%%%%%%%%%%%%%%%%%%%%%%%%%%%%%%%%%%%%%

\newpage
\section*{NeurIPS Paper Checklist}

%%% BEGIN INSTRUCTIONS %%%
The checklist is designed to encourage best practices for responsible machine learning research, addressing issues of reproducibility, transparency, research ethics, and societal impact. Do not remove the checklist: {\bf The papers not including the checklist will be desk rejected.} The checklist should follow the references and follow the (optional) supplemental material.  The checklist does NOT count towards the page
limit. 

Please read the checklist guidelines carefully for information on how to answer these questions. For each question in the checklist:
\begin{itemize}
    \item You should answer \answerYes{}, \answerNo{}, or \answerNA{}.
    \item \answerNA{} means either that the question is Not Applicable for that particular paper or the relevant information is Not Available.
    \item Please provide a short (1–2 sentence) justification right after your answer (even for NA). 
   % \item {\bf The papers not including the checklist will be desk rejected.}
\end{itemize}

{\bf The checklist answers are an integral part of your paper submission.} They are visible to the reviewers, area chairs, senior area chairs, and ethics reviewers. You will be asked to also include it (after eventual revisions) with the final version of your paper, and its final version will be published with the paper.

The reviewers of your paper will be asked to use the checklist as one of the factors in their evaluation. While "\answerYes{}" is generally preferable to "\answerNo{}", it is perfectly acceptable to answer "\answerNo{}" provided a proper justification is given (e.g., "error bars are not reported because it would be too computationally expensive" or "we were unable to find the license for the dataset we used"). In general, answering "\answerNo{}" or "\answerNA{}" is not grounds for rejection. While the questions are phrased in a binary way, we acknowledge that the true answer is often more nuanced, so please just use your best judgment and write a justification to elaborate. All supporting evidence can appear either in the main paper or the supplemental material, provided in appendix. If you answer \answerYes{} to a question, in the justification please point to the section(s) where related material for the question can be found.

IMPORTANT, please:
\begin{itemize}
    \item {\bf Delete this instruction block, but keep the section heading ``NeurIPS Paper Checklist"},
    \item  {\bf Keep the checklist subsection headings, questions/answers and guidelines below.}
    \item {\bf Do not modify the questions and only use the provided macros for your answers}.
\end{itemize} 
 

%%% END INSTRUCTIONS %%%


\begin{enumerate}

\item {\bf Claims}
    \item[] Question: Do the main claims made in the abstract and introduction accurately reflect the paper's contributions and scope?
    \item[] Answer: \answerTODO{} % Replace by \answerYes{}, \answerNo{}, or \answerNA{}.
    \item[] Justification: \justificationTODO{}
    \item[] Guidelines:
    \begin{itemize}
        \item The answer NA means that the abstract and introduction do not include the claims made in the paper.
        \item The abstract and/or introduction should clearly state the claims made, including the contributions made in the paper and important assumptions and limitations. A No or NA answer to this question will not be perceived well by the reviewers. 
        \item The claims made should match theoretical and experimental results, and reflect how much the results can be expected to generalize to other settings. 
        \item It is fine to include aspirational goals as motivation as long as it is clear that these goals are not attained by the paper. 
    \end{itemize}

\item {\bf Limitations}
    \item[] Question: Does the paper discuss the limitations of the work performed by the authors?
    \item[] Answer: \answerTODO{} % Replace by \answerYes{}, \answerNo{}, or \answerNA{}.
    \item[] Justification: \justificationTODO{}
    \item[] Guidelines:
    \begin{itemize}
        \item The answer NA means that the paper has no limitation while the answer No means that the paper has limitations, but those are not discussed in the paper. 
        \item The authors are encouraged to create a separate "Limitations" section in their paper.
        \item The paper should point out any strong assumptions and how robust the results are to violations of these assumptions (e.g., independence assumptions, noiseless settings, model well-specification, asymptotic approximations only holding locally). The authors should reflect on how these assumptions might be violated in practice and what the implications would be.
        \item The authors should reflect on the scope of the claims made, e.g., if the approach was only tested on a few datasets or with a few runs. In general, empirical results often depend on implicit assumptions, which should be articulated.
        \item The authors should reflect on the factors that influence the performance of the approach. For example, a facial recognition algorithm may perform poorly when image resolution is low or images are taken in low lighting. Or a speech-to-text system might not be used reliably to provide closed captions for online lectures because it fails to handle technical jargon.
        \item The authors should discuss the computational efficiency of the proposed algorithms and how they scale with dataset size.
        \item If applicable, the authors should discuss possible limitations of their approach to address problems of privacy and fairness.
        \item While the authors might fear that complete honesty about limitations might be used by reviewers as grounds for rejection, a worse outcome might be that reviewers discover limitations that aren't acknowledged in the paper. The authors should use their best judgment and recognize that individual actions in favor of transparency play an important role in developing norms that preserve the integrity of the community. Reviewers will be specifically instructed to not penalize honesty concerning limitations.
    \end{itemize}

\item {\bf Theory assumptions and proofs}
    \item[] Question: For each theoretical result, does the paper provide the full set of assumptions and a complete (and correct) proof?
    \item[] Answer: \answerTODO{} % Replace by \answerYes{}, \answerNo{}, or \answerNA{}.
    \item[] Justification: \justificationTODO{}
    \item[] Guidelines:
    \begin{itemize}
        \item The answer NA means that the paper does not include theoretical results. 
        \item All the theorems, formulas, and proofs in the paper should be numbered and cross-referenced.
        \item All assumptions should be clearly stated or referenced in the statement of any theorems.
        \item The proofs can either appear in the main paper or the supplemental material, but if they appear in the supplemental material, the authors are encouraged to provide a short proof sketch to provide intuition. 
        \item Inversely, any informal proof provided in the core of the paper should be complemented by formal proofs provided in appendix or supplemental material.
        \item Theorems and Lemmas that the proof relies upon should be properly referenced. 
    \end{itemize}

    \item {\bf Experimental result reproducibility}
    \item[] Question: Does the paper fully disclose all the information needed to reproduce the main experimental results of the paper to the extent that it affects the main claims and/or conclusions of the paper (regardless of whether the code and data are provided or not)?
    \item[] Answer: \answerTODO{} % Replace by \answerYes{}, \answerNo{}, or \answerNA{}.
    \item[] Justification: \justificationTODO{}
    \item[] Guidelines:
    \begin{itemize}
        \item The answer NA means that the paper does not include experiments.
        \item If the paper includes experiments, a No answer to this question will not be perceived well by the reviewers: Making the paper reproducible is important, regardless of whether the code and data are provided or not.
        \item If the contribution is a dataset and/or model, the authors should describe the steps taken to make their results reproducible or verifiable. 
        \item Depending on the contribution, reproducibility can be accomplished in various ways. For example, if the contribution is a novel architecture, describing the architecture fully might suffice, or if the contribution is a specific model and empirical evaluation, it may be necessary to either make it possible for others to replicate the model with the same dataset, or provide access to the model. In general. releasing code and data is often one good way to accomplish this, but reproducibility can also be provided via detailed instructions for how to replicate the results, access to a hosted model (e.g., in the case of a large language model), releasing of a model checkpoint, or other means that are appropriate to the research performed.
        \item While NeurIPS does not require releasing code, the conference does require all submissions to provide some reasonable avenue for reproducibility, which may depend on the nature of the contribution. For example
        \begin{enumerate}
            \item If the contribution is primarily a new algorithm, the paper should make it clear how to reproduce that algorithm.
            \item If the contribution is primarily a new model architecture, the paper should describe the architecture clearly and fully.
            \item If the contribution is a new model (e.g., a large language model), then there should either be a way to access this model for reproducing the results or a way to reproduce the model (e.g., with an open-source dataset or instructions for how to construct the dataset).
            \item We recognize that reproducibility may be tricky in some cases, in which case authors are welcome to describe the particular way they provide for reproducibility. In the case of closed-source models, it may be that access to the model is limited in some way (e.g., to registered users), but it should be possible for other researchers to have some path to reproducing or verifying the results.
        \end{enumerate}
    \end{itemize}


\item {\bf Open access to data and code}
    \item[] Question: Does the paper provide open access to the data and code, with sufficient instructions to faithfully reproduce the main experimental results, as described in supplemental material?
    \item[] Answer: \answerTODO{} % Replace by \answerYes{}, \answerNo{}, or \answerNA{}.
    \item[] Justification: \justificationTODO{}
    \item[] Guidelines:
    \begin{itemize}
        \item The answer NA means that paper does not include experiments requiring code.
        \item Please see the NeurIPS code and data submission guidelines (\url{https://nips.cc/public/guides/CodeSubmissionPolicy}) for more details.
        \item While we encourage the release of code and data, we understand that this might not be possible, so “No” is an acceptable answer. Papers cannot be rejected simply for not including code, unless this is central to the contribution (e.g., for a new open-source benchmark).
        \item The instructions should contain the exact command and environment needed to run to reproduce the results. See the NeurIPS code and data submission guidelines (\url{https://nips.cc/public/guides/CodeSubmissionPolicy}) for more details.
        \item The authors should provide instructions on data access and preparation, including how to access the raw data, preprocessed data, intermediate data, and generated data, etc.
        \item The authors should provide scripts to reproduce all experimental results for the new proposed method and baselines. If only a subset of experiments are reproducible, they should state which ones are omitted from the script and why.
        \item At submission time, to preserve anonymity, the authors should release anonymized versions (if applicable).
        \item Providing as much information as possible in supplemental material (appended to the paper) is recommended, but including URLs to data and code is permitted.
    \end{itemize}


\item {\bf Experimental setting/details}
    \item[] Question: Does the paper specify all the training and test details (e.g., data splits, hyperparameters, how they were chosen, type of optimizer, etc.) necessary to understand the results?
    \item[] Answer: \answerTODO{} % Replace by \answerYes{}, \answerNo{}, or \answerNA{}.
    \item[] Justification: \justificationTODO{}
    \item[] Guidelines:
    \begin{itemize}
        \item The answer NA means that the paper does not include experiments.
        \item The experimental setting should be presented in the core of the paper to a level of detail that is necessary to appreciate the results and make sense of them.
        \item The full details can be provided either with the code, in appendix, or as supplemental material.
    \end{itemize}

\item {\bf Experiment statistical significance}
    \item[] Question: Does the paper report error bars suitably and correctly defined or other appropriate information about the statistical significance of the experiments?
    \item[] Answer: \answerTODO{} % Replace by \answerYes{}, \answerNo{}, or \answerNA{}.
    \item[] Justification: \justificationTODO{}
    \item[] Guidelines:
    \begin{itemize}
        \item The answer NA means that the paper does not include experiments.
        \item The authors should answer "Yes" if the results are accompanied by error bars, confidence intervals, or statistical significance tests, at least for the experiments that support the main claims of the paper.
        \item The factors of variability that the error bars are capturing should be clearly stated (for example, train/test split, initialization, random drawing of some parameter, or overall run with given experimental conditions).
        \item The method for calculating the error bars should be explained (closed form formula, call to a library function, bootstrap, etc.)
        \item The assumptions made should be given (e.g., Normally distributed errors).
        \item It should be clear whether the error bar is the standard deviation or the standard error of the mean.
        \item It is OK to report 1-sigma error bars, but one should state it. The authors should preferably report a 2-sigma error bar than state that they have a 96\% CI, if the hypothesis of Normality of errors is not verified.
        \item For asymmetric distributions, the authors should be careful not to show in tables or figures symmetric error bars that would yield results that are out of range (e.g. negative error rates).
        \item If error bars are reported in tables or plots, The authors should explain in the text how they were calculated and reference the corresponding figures or tables in the text.
    \end{itemize}

\item {\bf Experiments compute resources}
    \item[] Question: For each experiment, does the paper provide sufficient information on the computer resources (type of compute workers, memory, time of execution) needed to reproduce the experiments?
    \item[] Answer: \answerTODO{} % Replace by \answerYes{}, \answerNo{}, or \answerNA{}.
    \item[] Justification: \justificationTODO{}
    \item[] Guidelines:
    \begin{itemize}
        \item The answer NA means that the paper does not include experiments.
        \item The paper should indicate the type of compute workers CPU or GPU, internal cluster, or cloud provider, including relevant memory and storage.
        \item The paper should provide the amount of compute required for each of the individual experimental runs as well as estimate the total compute. 
        \item The paper should disclose whether the full research project required more compute than the experiments reported in the paper (e.g., preliminary or failed experiments that didn't make it into the paper). 
    \end{itemize}
    
\item {\bf Code of ethics}
    \item[] Question: Does the research conducted in the paper conform, in every respect, with the NeurIPS Code of Ethics \url{https://neurips.cc/public/EthicsGuidelines}?
    \item[] Answer: \answerTODO{} % Replace by \answerYes{}, \answerNo{}, or \answerNA{}.
    \item[] Justification: \justificationTODO{}
    \item[] Guidelines:
    \begin{itemize}
        \item The answer NA means that the authors have not reviewed the NeurIPS Code of Ethics.
        \item If the authors answer No, they should explain the special circumstances that require a deviation from the Code of Ethics.
        \item The authors should make sure to preserve anonymity (e.g., if there is a special consideration due to laws or regulations in their jurisdiction).
    \end{itemize}


\item {\bf Broader impacts}
    \item[] Question: Does the paper discuss both potential positive societal impacts and negative societal impacts of the work performed?
    \item[] Answer: \answerTODO{} % Replace by \answerYes{}, \answerNo{}, or \answerNA{}.
    \item[] Justification: \justificationTODO{}
    \item[] Guidelines:
    \begin{itemize}
        \item The answer NA means that there is no societal impact of the work performed.
        \item If the authors answer NA or No, they should explain why their work has no societal impact or why the paper does not address societal impact.
        \item Examples of negative societal impacts include potential malicious or unintended uses (e.g., disinformation, generating fake profiles, surveillance), fairness considerations (e.g., deployment of technologies that could make decisions that unfairly impact specific groups), privacy considerations, and security considerations.
        \item The conference expects that many papers will be foundational research and not tied to particular applications, let alone deployments. However, if there is a direct path to any negative applications, the authors should point it out. For example, it is legitimate to point out that an improvement in the quality of generative models could be used to generate deepfakes for disinformation. On the other hand, it is not needed to point out that a generic algorithm for optimizing neural networks could enable people to train models that generate Deepfakes faster.
        \item The authors should consider possible harms that could arise when the technology is being used as intended and functioning correctly, harms that could arise when the technology is being used as intended but gives incorrect results, and harms following from (intentional or unintentional) misuse of the technology.
        \item If there are negative societal impacts, the authors could also discuss possible mitigation strategies (e.g., gated release of models, providing defenses in addition to attacks, mechanisms for monitoring misuse, mechanisms to monitor how a system learns from feedback over time, improving the efficiency and accessibility of ML).
    \end{itemize}
    
\item {\bf Safeguards}
    \item[] Question: Does the paper describe safeguards that have been put in place for responsible release of data or models that have a high risk for misuse (e.g., pretrained language models, image generators, or scraped datasets)?
    \item[] Answer: \answerTODO{} % Replace by \answerYes{}, \answerNo{}, or \answerNA{}.
    \item[] Justification: \justificationTODO{}
    \item[] Guidelines:
    \begin{itemize}
        \item The answer NA means that the paper poses no such risks.
        \item Released models that have a high risk for misuse or dual-use should be released with necessary safeguards to allow for controlled use of the model, for example by requiring that users adhere to usage guidelines or restrictions to access the model or implementing safety filters. 
        \item Datasets that have been scraped from the Internet could pose safety risks. The authors should describe how they avoided releasing unsafe images.
        \item We recognize that providing effective safeguards is challenging, and many papers do not require this, but we encourage authors to take this into account and make a best faith effort.
    \end{itemize}

\item {\bf Licenses for existing assets}
    \item[] Question: Are the creators or original owners of assets (e.g., code, data, models), used in the paper, properly credited and are the license and terms of use explicitly mentioned and properly respected?
    \item[] Answer: \answerTODO{} % Replace by \answerYes{}, \answerNo{}, or \answerNA{}.
    \item[] Justification: \justificationTODO{}
    \item[] Guidelines:
    \begin{itemize}
        \item The answer NA means that the paper does not use existing assets.
        \item The authors should cite the original paper that produced the code package or dataset.
        \item The authors should state which version of the asset is used and, if possible, include a URL.
        \item The name of the license (e.g., CC-BY 4.0) should be included for each asset.
        \item For scraped data from a particular source (e.g., website), the copyright and terms of service of that source should be provided.
        \item If assets are released, the license, copyright information, and terms of use in the package should be provided. For popular datasets, \url{paperswithcode.com/datasets} has curated licenses for some datasets. Their licensing guide can help determine the license of a dataset.
        \item For existing datasets that are re-packaged, both the original license and the license of the derived asset (if it has changed) should be provided.
        \item If this information is not available online, the authors are encouraged to reach out to the asset's creators.
    \end{itemize}

\item {\bf New assets}
    \item[] Question: Are new assets introduced in the paper well documented and is the documentation provided alongside the assets?
    \item[] Answer: \answerTODO{} % Replace by \answerYes{}, \answerNo{}, or \answerNA{}.
    \item[] Justification: \justificationTODO{}
    \item[] Guidelines:
    \begin{itemize}
        \item The answer NA means that the paper does not release new assets.
        \item Researchers should communicate the details of the dataset/code/model as part of their submissions via structured templates. This includes details about training, license, limitations, etc. 
        \item The paper should discuss whether and how consent was obtained from people whose asset is used.
        \item At submission time, remember to anonymize your assets (if applicable). You can either create an anonymized URL or include an anonymized zip file.
    \end{itemize}

\item {\bf Crowdsourcing and research with human subjects}
    \item[] Question: For crowdsourcing experiments and research with human subjects, does the paper include the full text of instructions given to participants and screenshots, if applicable, as well as details about compensation (if any)? 
    \item[] Answer: \answerTODO{} % Replace by \answerYes{}, \answerNo{}, or \answerNA{}.
    \item[] Justification: \justificationTODO{}
    \item[] Guidelines:
    \begin{itemize}
        \item The answer NA means that the paper does not involve crowdsourcing nor research with human subjects.
        \item Including this information in the supplemental material is fine, but if the main contribution of the paper involves human subjects, then as much detail as possible should be included in the main paper. 
        \item According to the NeurIPS Code of Ethics, workers involved in data collection, curation, or other labor should be paid at least the minimum wage in the country of the data collector. 
    \end{itemize}

\item {\bf Institutional review board (IRB) approvals or equivalent for research with human subjects}
    \item[] Question: Does the paper describe potential risks incurred by study participants, whether such risks were disclosed to the subjects, and whether Institutional Review Board (IRB) approvals (or an equivalent approval/review based on the requirements of your country or institution) were obtained?
    \item[] Answer: \answerTODO{} % Replace by \answerYes{}, \answerNo{}, or \answerNA{}.
    \item[] Justification: \justificationTODO{}
    \item[] Guidelines:
    \begin{itemize}
        \item The answer NA means that the paper does not involve crowdsourcing nor research with human subjects.
        \item Depending on the country in which research is conducted, IRB approval (or equivalent) may be required for any human subjects research. If you obtained IRB approval, you should clearly state this in the paper. 
        \item We recognize that the procedures for this may vary significantly between institutions and locations, and we expect authors to adhere to the NeurIPS Code of Ethics and the guidelines for their institution. 
        \item For initial submissions, do not include any information that would break anonymity (if applicable), such as the institution conducting the review.
    \end{itemize}

\item {\bf Declaration of LLM usage}
    \item[] Question: Does the paper describe the usage of LLMs if it is an important, original, or non-standard component of the core methods in this research? Note that if the LLM is used only for writing, editing, or formatting purposes and does not impact the core methodology, scientific rigorousness, or originality of the research, declaration is not required.
    %this research? 
    \item[] Answer: \answerTODO{} % Replace by \answerYes{}, \answerNo{}, or \answerNA{}.
    \item[] Justification: \justificationTODO{}
    \item[] Guidelines:
    \begin{itemize}
        \item The answer NA means that the core method development in this research does not involve LLMs as any important, original, or non-standard components.
        \item Please refer to our LLM policy (\url{https://neurips.cc/Conferences/2025/LLM}) for what should or should not be described.
    \end{itemize}

\end{enumerate}


\end{document}